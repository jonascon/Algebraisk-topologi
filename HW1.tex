\input{preamble.tex}
\usepackage[T1]{fontenc}

%\usepackage{fouriernc}

\usepackage{thmtools}
\usepackage{fancyhdr}
\pagestyle{fancy}
\usepackage{csquotes}
\newsavebox{\myheadbox}
\fancyhf{}
%\begin{flushright}
\lhead{
Jonas Conneryd \\ \url{conneryd@kth.se}}
%\end{flushright}
\rhead{970731-7559  \\
\the\year
}
\chead{\Large{\textbf{\textsf{Homework 1} \\ \vspace{-4pt}\textsf{\normalsize MM8042 Algebraic Topology}}}}
\cfoot{\thepage}
\declaretheoremstyle[headfont=\bfseries\sffamily]{normalhead}
\interfootnotelinepenalty=10000
%\title{\vspace{-2cm}\textbf{\textsf{ Homework 3}}}
\date{}
%\author{Jonas Conneryd \\
%conneryd@kth.se \\ 970731-7559}

\newcommand{\Set}{\mathsf{Set}}
\renewcommand{\C}{\mathsf{C}}
\begin{document}
%\maketitle
\thispagestyle{fancy}
\theoremstyle{normalhead}
\newtheorem{problem}{Problem}
\newtheorem{lemma}{Lemma}

\begin{problem} 
\textbf{(O)} Let $\Set$ be the category of sets and functions between them. Construct a functor $F$ from $\Set$ to itself with the following property: there exists an injective function between sets, call it $\alpha$, for which $F$ is not injective. In other words $F$ does not not preserve all injective functions. 
\end{problem}

\begin{proof}[Solution]
Trivial functor? Send all functions to some function?
\end{proof}

\newpage



\begin{problem}
\textbf{(O)} In class we defined a $\Delta$-complex $X_\bullet$ as a sequence of sets $X_0, X_1 \ldots, X_n, \ldots$ together with a collection of functions between them. More precisely, for every order-preserving function $\alpha: [m] \to [n]$, there is a function $\alpha^*: X_n \to X_m$, such that $(\alpha \circ \beta)^* = \beta^* \circ \alpha^*$. We then defined the geometric realization of $X_\bullet$ to be the quotient space of the disjoint union 
\[
\coprod_{n=0}^\infty X_n \times \Delta^n
\]
by the relations $(\alpha^*(x_n), \bar{u}_m) \sim (x_n, \alpha_*(\bar{u}_m))$, where $\alpha: [m] \to [n]$ is any order-preserving injective function, $x_n \in X_n$ and $\bar{u}_m \in \Delta^m$. Finally, we defined a $\Delta$-structure on a topological space $Z$ to be a homeomorphism of $Z$ with the geometric realization of some $\Delta$-complex. Is this equivalent to Hatcher's definition?
\end{problem}

\begin{proof}[Solution]
In Hatcher, a $\Delta$-complex structure on a space $X$ is a collection of maps $\sigma_\alpha: \Delta^n \to X$, where $n$ depends on $\alpha$, such that 
\begin{enumerate}
    \item The restriction $\sigma_\alpha \mid \mathring{\Delta}^n = \sigma_\alpha \mid \Delta^n - \partial \Delta^n$ is injective and each point of $X$ is in the image of exactly one such restriction.
    \item Each restriction of $\sigma_\alpha$ to a face of $\Delta^n$ is one of the maps $\sigma_\beta : \Delta^{n-1} \to X$. (Identifying the face of $\Delta ^n$ with $\Delta ^{n-1}$ by the canonical linear homeomorphism that preserves vertex ordering). 
    \item A set $A\subset X$ is open iff $\sigma_{\alpha}^{-1}(A)$ is open in $\Delta^{n}$ for each $\sigma_\alpha$. 
\end{enumerate}

Take an element $z \in Z$, where $Z$ is a topological space with $\Delta$-structure induced by the homeomorphism $\phi: G \to Z$, where $G$ is a geometric realization of the $\Delta$-complex $X_\bullet$. The open sets $U \in G$ are the images of the quotient map $q$ which identifies $(\alpha^*(x_n), \bar{u}_m) \sim  (x_n, \alpha_*(\bar{u}_m))$, and the open sets in $Z$ with the $\Delta$-structure are precisely $\phi(U)$, since $\phi$ is a homeomorphism. We can identify the maps $\sigma_\alpha$ in Hatcher's definition with the maps 
\end{proof}

\newpage



\begin{problem}
\textbf{(O)} Let $X$ be the quotient space of the 3-simplex $[v_0, v_1, v_2, v_3]$ by the relation that identifies the edge $[v_0, v_1]$ with $[v_2, v_3]$ and the edge $[v_0, v_2]$ with $[v_1, v_3]$  using the canonical order-preserving homeomorphism. Prove that $X$ deformation retracts onto the torus. 
\end{problem}

\begin{proof}[Solution]
\hfill
\begin{enumerate}[font=\normalfont,label=\textbf{(\alph*)}, wide]
\item
\end{enumerate}
\end{proof}

\newpage



\begin{problem}
\textbf{(W)} Suppose that $X_\bullet$ is a finite $\Delta$-complex. Define the \emph{Euler characteristic} of $X_\bullet$ to be the alternating sum $\chi(X_\bullet) = \# X_0-\#X_1+\#X_2-\ldots$. Though it is not obvious from the definition, one can show that all different $\Delta$-structures on the same space will have the same Euler characteristic. Find two different $\Delta$-structures on the torus and verify that they give the same Euler characteristic.
\end{problem}

\begin{proof}[Solution]
\hfill
\begin{enumerate}[font=\normalfont,label=\textbf{(\alph*)}, wide]
\item
\end{enumerate}
\end{proof}

\newpage


\begin{problem}
\textbf{(W)} Let $\mathsf{C}$ be a category, and let $\alpha: x\to y$ be a morphism in $\C$. We say that $\alpha$ is an \emph{isomorphism} if there exists another morphism $\beta: y \to x$ such that $\alpha \circ \beta = \mathsf{1}_y$ and $\beta \circ  \alpha  = \mathsf{1}_x$.
\begin{enumerate}[font=\normalfont,label=\textbf{(\alph*)}]
\item Prove that if such $\beta$ exists, it is unique. We will therefore write $\alpha^{-1}$ for this morphism. 

\item Let $\mathsf{Aut}(x)$ be the set of all isomorphisms of $x$ with itself. Prove that $\mathsf{Aut}(x)$ is a group.

\item Prove that if there is an isomorphism of objects $x$ and $y$ then there is an isomorphism of groups  $\mathsf{Aut}(x) \cong \mathsf{Aut}(y)$. 
\end{enumerate}
\end{problem}

\begin{proof}[Solution]
\hfill
\begin{enumerate}[font=\normalfont,label=\textbf{(\alph*)}, wide]
\item Let $\alpha: x\to y$ be a morphism, and suppose there are two maps $\beta_1, \beta_2: y \to x$ that make $\alpha$ into an isomorphism. Then we have
\[
\begin{aligned}
\alpha \circ \beta_1 &= \alpha \circ \beta_2 = 1_y, \\
 \beta_1  \circ \alpha&= \beta_2 \circ \alpha  = 1_x.
\end{aligned}
\]
This implies that
\[
\begin{aligned}
\beta_2 \circ ( \alpha \circ \beta_1) &= \beta_2 \circ ( \alpha \circ \beta_2) = \beta_2 \circ 1_y, \\
(\beta_1  \circ \alpha) \circ \beta_2 &= (\beta_2 \circ \alpha) \circ \beta_2  = 1_x \circ \beta_2.
\end{aligned}
\]
By associativity of morphism composition (which is guaranteed by the definition of a morphism) and using that $\beta_i \circ 1_y = 1_x \circ \beta_i = \beta_i$, we get
\[
\beta_2 = \beta_2 \circ( \alpha \circ \beta_1) =  (\beta_2 \circ \alpha) \circ \beta_1 = \beta_1, 
\]
so $\beta$ is unique.

\item We need to check closure, identity, inverse and associativity. An identity element for $\mathsf{Aut}(x)$ is provided by $1_x$, which is an isomorphism with inverse $1_x$. Associativity is provided by associativity of morphism composition, which holds by definition. An inverse for the isomorphism $\alpha: x \to x$ is provided by $\beta: x \to x$, where $\beta$ is as in the definition of isomorphism in the problem formulation. $\beta$ is itself an isomorphism with inverse $\alpha$, so it is in $\mathsf{Aut}(x)$. Finally, closure holds because for isomorphisms $\alpha_1, \alpha_2$ with inverses $\beta_1, \beta_2$ respectively, the composed map $\alpha_1 \circ \alpha_2$ is an isomorphism with inverse $\beta_2 \circ \beta_1$, where we have again invoked associativity of morphism composition.


\item Suppose we have an isomorphism $\alpha: x \to y$ with inverse $\beta: y \to x$. 
\end{enumerate}
\end{proof}




\end{document}