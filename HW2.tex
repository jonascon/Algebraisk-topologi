\input{preamble.tex}
\usepackage[T1]{fontenc}
\usepackage{tikz-cd}
%\usepackage{fouriernc}

\usepackage{thmtools}
\usepackage{fancyhdr}

\usepackage{csquotes}
\newsavebox{\myheadbox}
\fancypagestyle{normalpage}
{
%\begin{flushright}
\lhead{
Jonas Conneryd
}
%\end{flushright}
\rhead{\url{conneryd@kth.se}}
\chead{MM8042 Algebraic Topology}
\cfoot{\thepage}
}
\fancyhf{}
\fancypagestyle{firstpage}
{
%\begin{flushright}
\lhead{
Jonas Conneryd \\ \url{conneryd@kth.se}}
%\end{flushright}
\rhead{970731-7559  \\
\the\year
}
\chead{\Large{\scshape{\textbf{Homework 2} \\ \vspace{-4pt}\normalsize\textbf{ MM8042 Algebraic Topology}}}}
\cfoot{\thepage}
}

\pagestyle{normalpage}

\declaretheoremstyle[headfont=\bfseries\sffamily]{normalhead}
\interfootnotelinepenalty=10000
%\title{\vspace{-2cm}\textbf{\textsf{ Homework 3}}}
\date{}
%\author{Jonas Conneryd \\
%conneryd@kth.se \\ 970731-7559}

\newcommand{\Set}{\mathsf{Set}}
\renewcommand{\C}{\mathsf{C}}
\begin{document}
%\maketitle
\thispagestyle{firstpage}
\theoremstyle{normalhead}
\newtheorem{problem}{Problem}
\newtheorem{lemma}{Lemma}



\begin{problem}
\textbf{(O)} Find a way of identifying pairs of faces of $\Delta^3$ to produce a $\Delta$-complex structure on $S^3$ having a single 3-simplex, and compute the simplicial homology groups of this $\Delta$-complex. Hint: You want the quotient of the boundary of $\Delta^3$
by your identifications to be contractible.
\end{problem}
\begin{proof}[Solution]

\end{proof}


\newpage

\begin{problem}
\textbf{(O)} Does a level-wise injective chain homomorphism induce an injective homomorphism on homology? What about level-wise surjective homomorphisms?
\end{problem}

\begin{proof}[Solution]
A counterexample for injectivity is considering a level-wise injective chain homomorphism $C^{\Delta}_*(\Delta^1) \to C^{\Delta}_*(\Delta^2)$; We have



since the first homology group of $S^2$ is zero.
\end{proof}

\newpage

\begin{problem}
\textbf{(O)} Prepare a short presentation explaining the relationship between the fundamental group and the first homology group: the statement and an outline of the proof.
\end{problem}

\begin{proof}[Solution]

\end{proof}
\end{document}